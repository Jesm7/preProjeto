\section{REFERENCIAL TEÓRICO}

O texto a seguir constitui todo o embasamento teórico para a realização deste trabalho de conclusão de curso, que tem como objetivo desenvolver um dispositivo eletrônico acessível para monitorar a frequência cardíaca, saturação de oxigênio e temperatura corporal.

\subsection{Frequência cardíaca (Fc)}

As doenças cardiovasculares (DCV) são as principais causas de mortes no mundo, sendo um grave problema de saúde pública (Mendonça 2016). Segundo a Organização Mundial de Saúde (OMS), cerca de 17,9 milhões de pessoas perderam a vida devido a complicações cardiovasculares em 2019, representando aproximadamente 32\% das mortes globais.
Frequência cardíaca ou ritmo cardíaco é o número de vezes que o coração bate por minuto (bpm). Esse batimento pode ser dividido em várias fases – ciclo cardíaco (Ferreira, Almeida et al. 2007). Os batimentos cardíacos normais variam conforme a idade e o estado de saúde da pessoa, as tabelas a seguir mostram os parâmetros:

\subsection{Saturação (Sp02)}

\subsection{Temperatura corporal (Tc)}

\subsection{Arduino}

\subsubsection{Tipos de Arduino}

\subsubsubsection{Arduino UNO}

\subsubsubsection{Arduino Nano}

\subsubsubsection{Arduino Mega}

\subsection{Sensor MAX30102}

\subsection{Sensor MLX90614}

\subsection{Display Oled}

\subsection{Modulo GSM SIM800L}