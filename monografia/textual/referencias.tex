\section*{REFERÊNCIAS BIBLIOGRÁFICAS}

% Para escrever corretamente as referências bibliográficas e citações, recomendamos o site
% https://more.ufsc.br/

\\

\singlespacing 
\noindent EXCELEASY. Gráfico de linhas no Excel. Disponível em: https://exceleasy.com.br/grafico-de-linhas-no-excel/grafico-de-linhas-no-excel/ Acesso em: fev. 2024.

\\

\singlespacing
\noindent TABLES GENERATOR. Disponível em: https://www.tablesgenerator.com/ Acesso em: fev. 2024.
\\
% No manual de normalização de trabalhos acadêmicos do CEFET, temos a seguinte lista de Referências usadas como exemplo

\noindent BRASIL. Congresso Nacional. Lei nº 14.040, de 18 de agosto de 2020. Estabelece normas educacionais excepcionais a serem adotadas durante o estado de calamidade pública reconhecido pelo Decreto Legislativo nº 6, de 20 de março de 2020; e altera a Lei nº 11.947, de 16 de junho de 2009. Diário Oficial da União. Poder Legislativo, Brasília, 19 ago. 2020. Edição: 159, Seção 1, p. 4. Disponível em: https://www.in.gov. br/en/web/dou/-/lei-n-14.040-de-18-de-agosto-de-2020-272981525. Acesso em: 16 mar. 2021.

\singlespacing{
\noindent CHAUÍ, Marilena de Souza. \textbf{Convite à filosofia.} 13. ed. São Paulo: Ática, 2009. 424 p. }
\\

\singlespacing
\noindent DEMO, Pedro. \textbf{Conhecimento moderno}: sobre ética e intervenção do conhecimento. 3. ed. Petrópolis: Vozes, 1999, 317 p. 
\\

\singlespacing
\noindent \rule{2cm}{0.10mm}\textbf{ Metodologia do conhecimento científico.} São Paulo: Atlas, 2000, 216 p.
\\

\singlespacing
\noindent \rule{2cm}{0.10mm} \textbf{A educação do futuro e o futuro da educação}. Campinas, SP: Autores Associados, 2005, 191 p. (Coleção educação contemporânea). 
\\

\singlespacing
\noindent ELIAS, Norbert. A sociedade dos indivíduos. Rio de Janeiro: Jorge Zahar, 1994. 201 p. 
\\

\singlespacing
\noindent GEERTZ, Clifford. \textbf{A interpretação das culturas}. 13. reimpr. Rio de janeiro: LTC, 1989. 213 p. 
\\

\singlespacing
\noindent GIL, Antônio Carlos. \textbf{Como elaborar projetos de pesquisa}. 5. ed. São Paulo: Atlas, 2010. xiv, 184 p. 
\\

\singlespacing
\noindent LARAIA, Roque de Barros. \textbf{Cultura}: um conceito antropológico, 14. ed. Rio de Janeiro: Zahar, 2001. 
\\

\singlespacing
\noindent LATOUR, Bruno. \textbf{Ciência em ação}: como seguir cientistas e engenheiros sociedade afora. Tradução de Ivone C. Benedetti. 2. ed. São Paulo: Ed. UNESP, 2011. 422 p. 
\\

\singlespacing
\noindent MARCONI, Marina de Andrade; LAKATOS, Eva Maria. \textbf{Fundamentos de metodologia científica}. 7. ed. São Paulo: Atlas, 2010. xvi, 297 p. 
\\

\singlespacing
\noindent MEDEIROS, João Bosco. \textbf{Redação científica}: a prática de fichamentos, resumos, resenhas. 12. ed. São Paulo: Atlas, 2014. 331 p. 
\\

\singlespacing
\noindent PINTO, Álvaro Vieira. \textbf{O conceito de tecnologia}. Rio de Janeiro: Contraponto, 2005. xiv, 531 p. v. 1. 
\\

\singlespacing
\noindent PINTO, Álvaro Vieira. \textbf{O conceito de tecnologia}. Rio de Janeiro: Contraponto, 2005. xii, 794 p. v. 2 
\\

\singlespacing
\noindent SPALDING, M.; RAUEN, C.; VASCONCELLOS, L. M. R. de; VEGIAN, M. R. da C.; MIRANDA, K. C.; BRESSANE, A.; SALGADO, M. A. C. Desafios e possibilidades para o ensino superior: uma experiência brasileira em tempos de COVID-19. \textbf{Research, Society and Development}, [S. l.], v. 9, n. 8, 2020.Disponível em: https://rsdjournal. org/index.php/rsd/article/view/5970. Acesso em 29 mar. 2021.