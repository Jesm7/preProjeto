\section*{REFERÊNCIAS BIBLIOGRÁFICAS}

% Para escrever corretamente as referências bibliográficas e citações, recomendamos o site
% https://more.ufsc.br/


\noindent Alencar, D. C., et al. (2013). "Sistema de monitoramento de temperatura corporal humana utilizando software embarcado e tempo real baseado em computação ubíqua." Proceedings of World Congresson Systems Engineering and Information Technology: 193-197.


\singlespacing
\noindent de Robótica, G. (2012). "Introduçao ao arduino." Notas de aula, Universidade Federal do Mato Grosso do Sul 10.

% No manual de normalização de trabalhos acadêmicos do CEFET, temos a seguinte lista de Referências usadas como exemplo
\singlespacing
\noindent Ferreira, F. G., et al. (2007). "Efeitos da ingestão de diferentes soluções hidratantes nos níveis de hidratação e na frequência cardíaca durante um exercício de natação intervalado." Revista Portuguesa de Ciências do Desporto, Brasília 7(3): 319-327.


\singlespacing
\noindent Lima, D. W. d. C. (2009). "Oxímetro de pulso com transmissão de sinal sem fios."


\singlespacing
\noindent Mendonça, V. F. (2016). "A Relação Entre o Sedentarismo, Sobrepeso e Obesidade com as Doenças Cardiovasculares em Jovens Adultos: uma Revisão da Literatura." Saúde e Desenvolvimento Humano 4(1): 79-90.


\singlespacing
\noindent Monteiro, F. A. M., et al. (2021). "DETECTOR DE QUEDAS VIA CELULAR." Revista Brasileira de Mecatrônica 4(1): 58-71.


\singlespacing
\noindent Nascimento Júnior, V. M. d. (2022). "Sistema de monitoramento remoto de batimentos cardíacos e oximetria utilizando sensores MAX30100 e módulos de aquisição com ESP8266."


\singlespacing
\noindent Ribeiro, M. P., et al. (1992). "Análise espectral da freqüência cardíaca. Conceitos básicos e aplicação clínica." Arq Bras Cardiol 59(3): 141-149


\singlespacing
\noindent RIBEIRO, T. J. d. S. (2018). Análise exergética do sistema térmico do corpo humano para avaliação de conforto térmico, Dissertação (Mestrado). Faculdade de Engenharia Mecânica, Universidade ….


\singlespacing
\noindent Soares, J. V. B., et al. (2022). SISTEMA DE MONITORAMENTO DE SINAIS BIOMÉDICOS (SMSB). Congresso Brasileiro de Automática-CBA.


\singlespacing
\noindent Zazeka, F. E. (2021). "Desenvolvimento de um dispositivo portátil para gerenciamento de dados na prevenção e gestão da COVID-19."


\singlespacing
\noindent META PLATFORMS, INC, Meta Open Source. React Native: Learn once, write anywhere. 2024. Disponível em: https://reactnative.dev/. Acessado em: Agosto, 2024


\singlespacing
\noindent GOOGLE, Google for Developers. Firebase. 2024. Disponível em: https://firebase.google.com/docs. Acessado em: Agosto, 2024
