% \section{RESULTADOS E DISCUSSÕES}
\section{RESULTADOS ESPERADOS}

Espera-se que o dispositivo seja capaz de ler com precisão, eficiência e efiácia os sinais vitais. Que os dados coletados sejam capazes de representar com veracidade a situação do usuário, de modo que essas variáveis sejam capazes de auxiliar em um alerta prévio de complicações. Com relação ao aplicativo, espera-se q a interface seja de fácil utilização e contemple todos os dados medidos, através de tabelas e/ou gráficos, de modo que o usuário não fique com dúvidas sobre o que está observando.
% \subsection{Exemplos de gráficos e figuras}

% Para inserir figuras no \LaTeX é utilizado o comando \textsc{includegraphics}. Para indexar essa figura precisamos do ambiente \textsc{figure}. Veja como exemplo a Figura \ref{fig:figura1}. Neste template, as figuras devem ser armazenadas fisicamente no diretório \textsc{graphics}.
% Ajuste a label e a legenda adequadamente.

% \begin{figure}   %\captionsetup{singlelinecheck=false,justification=justified,format= hang,font=normalsize}
%     \caption{Exemplo de figura.}
%     \vspace{-0.5 cm}
%     \begin{center}
%     \includegraphics[scale=0.8]{graphics/exemploFigura1.png}
%     \end{center}
%     \vspace{-0.3 cm}
%     \small{Fonte: EXCELEASY (2024).}
%     \label{fig:figura1}
% \end{figure}

% \subsection{Exemplo de tabelas}

% Pra Quadros ou Tabelas, recomenda-se  a ajuda do site ``Tables Generator'' (TABLES GENERATOR, 2024). Além disso, para indexação é necessário o ambiente \texttt{table}. Veja um exemplo na Tabela

% \begin{table}[h]
% \centering
% \caption{Exemplo de tabela.}
% \begin{tabular}{|l|l|l|l|}
% \hline
% & \textbf{A} & \textbf{B} & \textbf{C} \\
% \hline
% 1 & 11         & 12         & 13         \\
% 2 & 0          & 2          & 3          \\
% 3 & 25         & 26         & 27    \\     
% \hline    
% \end{tabular}
% \vspace{0.2 cm}

% \small{Fonte: A autora.}
% \end{table}

% \normalsize



% \subsection{Exemplo de Programas, Scripts ou Algoritmos}

% Programas ou Scripts podem ser escritos em \LaTeX usando o pacote \texttt{lstlisting}. É importante que esse pacote seja configurado adequadamente de acordo com a linguagem de programação utilizada. Como exemplo, veja o Programa que está escrito em Python. Neste template, a configuração do pacote está discponível no arquivo \texttt{estrutura.tex}.

% \begin{lstlisting}[caption={Função Eliminação de Gauss},label={prog:gauss},language=Python]
% import numpy as np

% def gauss(n, A, b):#algoritmo do livro do Ruggiero
%     '''
%     supor que o elemento que estah na posicao akk 
%     eh diferente de zero no inicio da etapa k
%     '''
%     A_copy = A.copy()
%     b_copy = b.copy()
%     for k in range(0,n-1):
%         for i in range(k+1,n):
%             m = float(A_copy[i][k]/A_copy[k][k])
%             for j in range(k,n):
%                 A_copy[i][j] = float(A_copy[i][j]-
%                     float(m*A_copy[k][j]))
%             b_copy[i] = float(b_copy[i]) - float(m*b_copy[k])
%             A_copy[i][k] = 0
%     #fase da resolucao
%     return retroativa(n,A_copy,b_copy)
% \end{lstlisting}
% \small{Fonte: A autora}

% \normalsize


%  \lipsum[1-2]



