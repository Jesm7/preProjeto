\newpage
\thispagestyle{empty}
\vspace{1.5cm}
\begin{center}
{\large{\textbf{RESUMO}}}
\end{center}
\vspace{0.5cm}

\begin{spacing}{1.0}
\noindent A dengue é a doença viral transmitida por um arbovírus que mais rapidamente se espalha no mundo. Seu principal vetor é o mosquito \emph{Aedes aegypti}, muito adaptado ao meio urbano. No caso do Brasil, já podemos observar o grande surto epidêmico em 2011, principalmente depois da entrada do sorotipo 4 no país. A incidência de dengue mostra uma clara dependência das variações sazonais. Além disso, fatores relacionados a cada indivíduo envolvido no processo podem ser analisados dentro do contexto epidêmico. Os modelos matemáticos têm sido cada vez mais utilizados para se tentar mapear os comportamentos das doenças infecciosas. Adicionado a eles, utiliza-se também de recursos computacionais para simulações agregando características comportamentais dos indivíduos e outros aspectos relevantes ao estudo. Para estudar a propagação da doença propomos um modelo matemático e computacional a partir de autômatos celulares, o qual procura identificar os fatores que contribuem para a proliferação da dengue. O foco principal do trabalho é a interação entre os indivíduos envolvidos no processo de espalhamento da dengue, levando em conta aspectos espaciais. Por meio da teoria de modelos baseados em indivíduos e com recursos computacionais, construimos um autômato celular, considerando as características comportamentais das epidemias de dengue - implementando o comportamento cíclico da doença, e as interações indivíduo a indivíduo numa região hipotética. O modelo nos permitiu capturar informações gerais sobre a dengue além de comportamentos essenciais dos indivíduos, afirmando a característica básica desse tipo de modelagem: a partir de regras simples, capturar informações complexas embutidas nas interações entre os agentes. Os resultados obtidos nos conferiram a importância das interações entre o vetor e o hospedeiro na permanência da doença. Apesar de tratarmos uma região hipotética obtivemos comportamentos semelhantes aos encontrados na natureza. As principais particularidades encontradas nos remetem às interações, densidades populacionais e, fundamentalmente, aos deslocamentos dos indivíduos e às elevadas taxas de casos assintomáticos da doença.
\end{spacing}

\vspace{1.5ex}

\noindent {\bf Palavras-chave}: Epidemias. Dengue. Modelos baseados em indivíduos. Autômatos celulares.

