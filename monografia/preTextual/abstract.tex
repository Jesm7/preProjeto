\newpage
\thispagestyle{empty}
\vspace{1.5cm}
\begin{center}
{\large{\textbf{ABSTRACT}}}
\end{center}
\vspace{0.5cm}
 \hyphenation{especially automata taking epidemics capture behaviors}
\begin{spacing}{1.0}
\noindent Dengue is a viral disease transmitted by an arbovirus that quickly spreads around the world. Its main vector is the mosquito \emph{Aedes aegypti} urban environment well adapted. In Brazil, it is already observed a large outbreak in 2011 especially after serotype 4 arrived in the country. Dengue's incidence shows a clear dependence on seasonal variations. In addition, factors related to each individual involved in the process can be analyzed within epidemic context. Mathematical models have been increasingly used to try to map infectious diseases behavior. Added to them, it is also used computing resources for simulations by aggregating individual behavioral characteristics and other research relevant aspects. To study disease spread, we propose a mathematical and computer model from cellular automata, which seeks to identify factors that contribute to dengue spread. The main focus of this paper is the interaction among spread dengue involved individuals taking into account spatial aspects. Through individual based modelling theory and with computational resources, build cellular automata, considering dengue epidemics behavioral characteristics - implementing the cyclical disease behavior and individual to individual interactions in a hypothetical region. The model allowed us to capture dengue general information as well as individuals' essential behaviors, claiming this type of modeling basic characteristic: from simple rules, capture complex information embedded in agents interactions. The results gave us the importance vector and host interactions to disease maintenance. Despite treating a hypothetical region the obtained behaviors were similar to those found in nature. The main found peculiarities are related to interactions, population densities and, crucially, to individuals movements and the high rates of disease asymptomatic cases.

\end{spacing}

\vspace{1.5ex}

\noindent {\bf Keywords}: Epidemics. Dengue. Individual-Based Modelling. Cellular automata.


