%%
%% Capítulo : GLOSSÁRIO
%%
\section*{\centering GLOSSÁRIO}

%\begin{center}
%    \large \textbf{ANEXO A - GLOSSÁRIO}
%\end{center}


Glossário de termos epidemiológicos apresentados no texto desta dissertação.

\begin{description}
\item[Anautógeno, mosquito:] É a espécie de mosquitos onde as fêmeas necessitam de alimentação sanguínea para adquirirem os aminoácidos necessários à produção dos ovos. Machos e fêmeas podem ainda alimentar-se de glicídios obtidos do néctar de flores e de suco de frutos e utilizá-los como combustível energético para o vôo e outras atividades.
\item[Anticorpo:] Proteína produzida no sangue de vertebrados após exposição a um antígeno. O anticorpo liga-se especificamente ao antígeno e assim estimula sua inativação por outros componentes do sistema imune. A maior classe de anticorpos são imunoglobulinas A ou IgA, encontrada predominantemente em secreções corporais tais como saliva. As IgM e IgG são tipicamente produzidas sequencialmente em resposta a infecções por microparasitas; IgE é frequentemente elevada na resposta a infecção helmíntica. Somente IgG é capaz de cruzar a placenta para conferir imunidade materna.
\item[Antígeno:] Proteína estranha ao organismo que elicita em resposta imune específicamente dirigida contra ela.

\item[Arbovírus:] Vírus que usam artrópodes como vetores a são transmitidos através da saliva destes para o hospedeiro definitivo. Ex.: dengue e febre amarela.

\item[Arbovirose:] v. arbovírus.

\item[Artrópode:] São animais invertebrados caracterizados por possuírem membros rígidos e articulados e terem vários pares de pernas. Compõem o maior filo de animais existentes, representados pelos gafanhotos (insetos), aranhas (aracnídea), caranguejos (crustáceos), centopeias (quilópodes) e embuás (diplópodes).

\item[Ciclo gonotrófico:] Ciclo vital do mosquito (ciclo completo de desenvolvimento ovárico do mosquito), ou seja, o tempo que decorre entre a ingestão de sangue (através da picada) e a postura dos ovos.

\item[Contato, taxa:] Taxa de contato entre susceptíveis e infectados. Medido em indivíduos por unidade de tempo.

\item[Endemia:] Termo usado para descrever níveis de infecção que não exibem grande flutuação temporal em um determinado lugar. Para microparasitas tais como sarampo, o termo é usado para indicar uma infecção que persiste em uma população por longo tempo sem necessidade de ser reintroduzida por uma fonte externa. Endemicidade estável é o termo utilizado para uma doença transmissível que não mostra tendência secular para aumentar ou diminuir.

\item[Epidemia:] Rápido aumento nos níveis de uma infecção, típico dos microparasitas (com imunidade de longa duração e curtos tempo de geração). Uma epidmia é anunciada por um aumento exponencial no número de casos no tempo e um subseqüente declínio devido ao esgotamento do número de susceptíveis. Uma epidemia pode se originar pela introdução de um novo patógeno (ou linhagem) numa população anteriormente não exposta ao mesmo, ou pelo mesmo patógeno como resultado do aumento no número de susceptíveis algum tempo após uma epidemia prévia.

\item[Expectativa de vida:] O mesmo que longevidade, o tempo médio de vida dos indivíduos em uma população.

\item[Força de infecção:] Taxa per capita de infecção de susceptíveis.

\item[Holometábolo:] Holometábolo é um animal que tem metamorfose completa durante o seu desenvolvimento.
Nos insetos holometábolos o desenvolvimento é tido da seguinte forma: ovo, larva, pupa jovem, pupa adiantada, imago.

\item[Horizontal, transmissão:] Transmissão ocorrendo dentro de uma população entre seus indivíduos, mas que não inclui transmissão vertical.

\item[Imunidade:] 1) Estado em que um hospedeiro não é susceptível à uma infecção ou doença; ou 2) o mecanismo pelo qual isto é alcançado. O indivíduo adquire imunidade através de uma das três rotas: imunidade natural ou inata, geneticamente herdada ou adquirida através de anticorpos maternos; imunidade adquirida, conferida após contato com a doença; e imunidade artificial, após vacinação bem sucedida (também imunidade específica ou resistência). A imunidade específica é dividida em imunidade celular, atuando via células T e imunidade humoral envolvendo anticorpos e células B.

\item[Incidência:] Taxa de aparecimento de casos novos numa população. Classicamente medida como "taxa de ataque".

\item[Infecção:] Replicação de um microparasita em seu hospedeiro, podendo haver ou não doença.

\item[Infeccioso, período:] Período de tempo durante o qual infectados são capazes de transmitir a infecção para qualquer hospederio susceptível ou vetor com os quais entre em contato. Note que o período infeccioso pode não ser necessariamente associado com sintomas da doença.

\item[Infectado:] Hospedeiro que tem uma infecção.

\item[Intermediário, hospedeiro:] V. vetor.

\item[Latente, período:] Período da infecção em que o indivíduo é infeccioso para os susceptíveis. Em helmintos, é chamado de período pré-patente. Não confundir com período de incubação.

\item[Limiar de transmissão:]  Ocorre quando a taxa reprodutiva básica R0 de um parasita é igual 1. Abaixo deste limiar a doença é incapaz de se manter na população. Para parasitas de transmissão direta há um limiar de transmissão para o tamanho da população hospedeira.

\item[Mortalidade, taxa:] Mortes per capita numa população. A taxa de mortalidade é a recíproca da expectativa de vida de uma população.

\item[Oligossintomático:] que apresenta pouco ou quase nenhum sintoma da doença.

\item[Ovariolar:] Relativo a ovário.

\item[Oviposição:] Ato do mosquito fêmea por ovos.

\item[Pandemia:] Epidemia de dimensões continentais.

\item[Período de incubação:] Tempo decorrido entre a infecção e o aparecimento dos sintomas de uma doença. Não confundir com período de latência.

\item[Portador:] Indivíduo infectado que não manifesta os sintomas da doença. Há dois tipos de estado portador: portadores silenciosos, que retém suainfecciosidade, e portadores latentes, que não são infecciosos. Por exemplo, muitos daqueles infectados com tuberculose são portadores silenciosos, enquanto a infecção pelo herpesvírus pode criar portadores latentes.

\item[Pupa:] Uma das fases do desenvolvimento de um inseto.

\item[Quiescência:] Período durante o qual uma infecção está presente, porém sem atividade dentro de um hospedeiro: p. ex., o período entre um ataque agudo de varicela e um subsequente recrudescência de zoster. Não é o mesmo que latência.

\item[Recrudescimento, recrudescência:] Reaparecimento de doença em um hospedeiro cuja infecção era quiescente.

\item[Repasto sanguíneo:] É o ato do inseto se alimentar de sangue diretamente do animal. No caso da dengue o repasto sanguíneo é feito pela fêmea do mosquito (vetor) que se alimenta de sangue humano pela picada.

\item[Resistência:] 1) Redução, devido a seleção genética, da susceptibilidade de um parasita ou seu vetor à quimioterapia. 2) Capacidade do hospedeiro em ressitir a um patógeno. Compare com imunidade.

\item[Sintoma:] 1) Condição somática relatada por um indivíduio sofrendo de uma doença. 2) Qualquer evidência num indivíduo infectado que leve a um diagnóstico ou identificação de uma infecção.

\item[Sorologia:] Estudo das reações antígeno-anticorpos. Via de regra, o uso de dados sorológicos para inferir sobre a história infecciosa pregressa de um indivíduo.

\item[Sorotipos:] 1) Variedade de anticorpos de um indivíduo, via de regra baseado em análises de amostras de sangue ou saliva. 2) Diferentes linhagens de um patógeno distinguidas pelos diferentes anticorpos que eles induzem no hospedeiro, ou com os quais reagem in vitro. Deste modo, a palavra sorotipo é também aplicada a uma linhagem particular, sendo este seu uso clínico mais comum. A variedade de anticorpos usada para definir um sorotipo depende obviamente daqueles que estão disponíveis para o pesquisador. Algumas vezes, como p. ex., para o sarampo, a presença de um anticorpo conhecido no soro de um indivíduo correlaciona muito bem com a observação clínica de que o indivíduo está protegido contra futuras infecções. Porém, algumas vezes, como, p. ex., para a malária, não há ainda uma relação definida entre um dado sorotipo e a presença de uma imunidade funcional, o que pode fazer a palavra sorotipo não ser útil quando se trata de distinguir entre diferentes parasitas com o propósito de se compreender suas transmissões.

\item[Suscetível:] Indivíduo acessível ou capaz de ser infectado por um patógeno.

\item[Taxa:] 1) Número de eventos ocorridos dividido pelo tempo em que eles aconteceram. 2) Variação na quantidade de algo pelo tempo usado para se medi-la.

\item[Taxa (bruta) de nascimento:] Número de nascidos vivos em um ano dividido pelo tamanho da população.

\item[Taxa (bruta) de mortalidade:] Número de mortes no ano dividido pelo tamanho da população.

\item[Taxa ou razão reprodutiva básica, R0:] Parâmetro adimensional que encapsula os detalhes biológicos envolvendo diferentes mecanismos de transmissão. Para os microparasitas, R0é definido como o número médio de casos secundários de infecção originados de um caso primário quando este, encontrando-se no seu período infeccioso, é introduzido numa população que consiste somente de indivíduos susceptíveis. Para macroparasitas, R0 é o número médio de descendentes de fêmeas (ou de toda descendência, tratando-se de espécies hermafroditas) produzidos durante o tempo de vida de um parasita fêmea maduro, que alcança sua maturidade reprodutiva na ausência de restrições densidade-dependente relativas à sobrevivência ou reprodução do parasita.

\item[Transmissão:] Processo pelo qual um patógeno passa de uma fonte de infecção para um novo hospedeiro. Há dois tipos de transmissão: horizontal e vertical. A maioria das formas de transmissão se dá horizontalmente, ou seja, de hospedeiro para hospedeiro.

\item[Transmissão vertical:] Transmissão vertical ocorre quando um genitor passa a infecção para seu feto, como ocorre na sífilis humana e entre artrópodes que transmite transovarianamente arbovírus. A infecção perinatal é uma forma especial de transmissão vertical.

\item[Vetor:] 1) Hospedeiro de parasitas com ciclos indiretos de vida. 2) Qualquer coisa que transmite parasitas. 3) Um invertebrado transmissor de vírus para vertebrados.

\item[Vetorial, capacidade:] Em infecções transmitidas por vetores tais como a malária, a capacidade vetorial é um conceito análogo à taxa de contato em doenças de transmissão direta. Isto é uma função da 1) densidade do vetor em relação ao seu hospedeiro vertebrado, 2) da frequência com que ele se alimenta de sangue da espécie hospedeira, 3) da duração do período latente no vetor, e 4) da expectativa de vida do vetor.

\item[Viremia:] Presença de vírus no sangue durante a evolução de processo infeccioso.

\item[Virulência:] 1) Taxa de mortalidade de uma infecção. 2) Grau de dano conferido pelo patógeno ao seu hospedeiro. Há diferentes usos para este conceito, porém, o que eles têm em comum é que eles se referem ao efeito de um hospedeiro infectado, não ao grau de transmissibilidade para um susceptível subsequente.

\end{description} 